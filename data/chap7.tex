\chapter{结论和展望}

在本文中我们简单总结了正则化在机器学习中的应用,并提出了结合时间和空间
正则化的思想,并将其应用到视频压缩这一具体问题上,有此给出了一个用于视
频压缩的统一的主动学习和半监督学习的框架。给定的算法的灵感来源于基于图
的半监督学习的最新进展 \cite{Manifold-Regularization-Journal}。使用最优
化实验设计的技术,我们通过最小化系数的协方差矩阵的方式来进行选点。对于
视频压缩来说,主动学习的方法将会用于选取最有代表性的像素点,而半监督学
习则会用于着色的过程。实验结果显示我们的方法的性能地超过了目前的其他方
法,显示了时空正则化在合适的数据上能对经典的机器学习带来性能提升。

我们希望这里提出的算法和框架可以提供一个图像和视频分析的新的视角,这篇
论文主要集中在视频压缩上,但是,相同的思想可以被应用到其他类似的问题
上,凡是数据带有时间局部性的问题,都可以尝试利用时空正则化来对已有的方
法进行进一步改进。

